% !TEX TS-program = pdflatex
% !TEX encoding = UTF-8 Unicode

\documentclass[a4paper]{article}

\usepackage[swedish]{babel}
\usepackage[T1]{fontenc}
\usepackage[utf8]{inputenc}
\usepackage[pdftex]{graphicx}
\usepackage{float}
\usepackage{fancyhdr}


\usepackage{booktabs} % for much better looking tables
\usepackage{array} % for better arrays (eg matrices) in maths
\usepackage{paralist} % very flexible & customisable lists (eg. enumerate/itemize, etc.)
\usepackage{verbatim} % adds environment for commenting out blocks of text & for better verbatim
\usepackage{subfig} % make it possible to include more than one captioned figure/table in a single float


%%% HEADERS & FOOTERS
\author{Jonathan Karlsson, Niclas Olofsson, Paul Nedstrand\\jonka293, nicol271, paune\\Grupp 2}
\pagestyle{fancy} % options: empty , plain , fancy
\renewcommand{\headrulewidth}{1pt} % customise the layout...
\fancyhead[LO,LE]{Jonathan, Niclas, Paul\\Rapportmall}
\lfoot{}\cfoot{\thepage}\rfoot{}

%%%% SECTION TITLE APPEARANCE
%\usepackage{sectsty}
%\allsectionsfont{\sffamily\mdseries\upshape} % (See the fntguide.pdf for font help)
%% (This matches ConTeXt defaults)
%
%%%% ToC (table of contents) APPEARANCE
%\usepackage[nottoc,notlof,notlot]{tocbibind} % Put the bibliography in the ToC
%\usepackage[titles,subfigure]{tocloft} % Alter the style of the Table of Contents
%\renewcommand{\cftsecfont}{\rmfamily\mdseries\upshape}
%\renewcommand{\cftsecpagefont}{\rmfamily\mdseries\upshape} % No bold!

%%% END Article customizations

%%% The "real" document content comes below...

\title{Labrapport 4\\ \vspace{2 mm} {\large TSEA44}}
%\date{} % Activate to display a given date or no date (if empty),
         % otherwise the current date is printed

\begin{document}
\maketitle

\tableofcontents
\newpage

\section{Inledning}
Det fjärde och sista miniprojektet i kursen, lab 4, handlar om att skapa en egen assembler-instruktion
för skrivning till minnet. Syftet med denna instruktion är att snabba upp koden för att
Huffman-koda den resulterade bilden från vår jpeg-accelerator.\\

Instruktionen tar en längd samt ett data som argument, och sparar all inkommen data
i en buffer tills minst 8 bitar erhållits. I detta fall skrivs den buffrade datan till minnet;
en byte åt gången och så många bytes som möjligt (maximalt två).\\


\section{Design}
\begin{itemize}
\item How does your hardware work?
\end{itemize}

\section{Resultat}
\subsection{Verifiering av hårdvarans funktion}
För att kontrollera att vår hårdvara fungerade, började vi med att anropa vår instruktion från det
montor-program som körs när datorn startar. Efter en del felsökning och justering övergick vi till att testa
koden på en FPGA med samma monitor-program. Vi använde monitorns inbyggda kommando för att visa minnesadresser
för att verifiera att rätt data skrevs till minnet. Ganska snabbt insåg vi behovet av att kunna felsöka
även i denna miljö utan att behöva syntetisera om koden varje gång, och skrev därför testprogrammet asm.c
som vi kunne ladda in i minnet och köra via monitorn.\\

Det största problem vi hade under denna lab, och även det svåraste vi fått under labkursen,
var att sista biten i varje byte vi skrev till minnet blev fel. Till skillnad från de tidigare fel
vi fått under kursen så fick vi varken några varningar av värde vid syntetiseringen, konstiga
odefinierade signaler eller märkliga läs/skrivcykler vid simulering.\\

Efter noggrannt studerande av syntesrapporten upptäckte vi att en felaktig ihopslagning av två bitar i ett
register orsakade problemet. Vi gjorde om koden för skrivning till det aktuella registret. Simuleringen blev
fortfarande likadan som tidigare, men ingen konstig ihopslagning gjordes vid syntetiseringen,
vilket löste vårt problem.\\

Till sist testades även hårdvaran genom att instruktionen användes av JPEG-acceleratorn. jchuff.c modifierades,
för att använda set bit-instruktionen för skrivning till minnet. Efter en rejäl stunds felsökande
genererades till sist en korrekt bild.\\

\subsection{Prestanda}
Vi använde vårt testprogram och den prestandaräknare som inkluderades i jpegtest-programmet för att mäta prestandan på
några olika slags anrop till vår set bit-instruktion (Tabell \ref{tab:sbit_performance}). Dels varierade vi storleken, dels testade vi även att enbart skriva 0xFF, vilket i vår implementation gör två minnesskrivningar.\\

\begin{table}[ht]
    \centering
    \begin{tabular}{l l l}
        Storlek &  Data  &   Antal klockcykler\\
        \hline
        2       &  0x00  &   10\\
        8       &  0x00  &   13\\
        8       &  0xFF  &   17\\
        16      &  0x00  &   17\\
    \end{tabular}
    \caption{ Antal klockcykler per anrop av vår sbit-instruktion. Tabellen visar medelvärdet av 100 försök. }
    \label{tab:sbit_performance}
\end{table}

Vid användning i JPEG-acceleratorn jämförde vi utskriften av prestandamätningen för en version av
programmet som kompilerades utan set bit-instruktionen (Tabell \ref{tab:jpeg_sw_performance}), mot resultatet med denna instruktion påslagen (Tabell \ref{tab:jpeg_sbit_performance}). Båda dessa versioner innehåller alla de tidigare förbättringar
som gjorts; hårdvaru-DCT samt DMA.

\begin{table}[ht]
    \centering
    \begin{tabular}{l l}
        Beskrivning & Antal klockcykler\\
        \hline
        Main program  & 25 847 242 \\
        Init  &  6 600 044 \\
        Encode\_image  & 19 247 198 \\
        Forward\_DCT  & 6 489 189 \\
        Copy  & 0 \\
        DCT kernel  & 0 \\
        Quantization  & 6 489 189 \\
        Huffman encoding  & 12 226 950 \\
        Emit\_bits  &  4 983 905 \\
    \end{tabular}
    \caption{ Prestanda för JPEG-acceleratorn utan sbit-instruktionen }
    \label{tab:jpeg_sw_performance}
\end{table}

\begin{table}[ht]
    \centering
    \begin{tabular}{l l}
        Beskrivning & Antal klockcykler\\
        \hline
        Main program  & 21 754 287 \\
        Init  & 6 653 818 \\
        Encode\_image  & 15 100 469 \\
        Forward\_DCT  & 6 396 157 \\
        Copy  & 0 \\
        DCT kernel  & 0 \\
        Quantization  & 6 396 157 \\
        Huffman encoding  & 8 126 873 \\
        Emit\_bits  & 1 306 129 \\
    \end{tabular}
    \caption{ Prestanda för JPEG-acceleratorn med sbit-instruktionen }
    \label{tab:jpeg_sbit_performance}
\end{table}

\subsection{FPGA-användning}
\begin{table}[ht]
    \centering
    \begin{tabular}{l l l}
        Flip Flops   &      7499 out of   46080  & 16\% \\
        4 input LUTs &      12519 out of  46080  & 28\% \\
        MULT18X18s   &      19 out of 120  &   15\% \\
        RAMB16s      &      42 out of 120  &   35\% \\
    \end{tabular}
    \caption{ De mest intressanta delarna ur syntes-rapporten för JPEG-acceleratorn med DCT, DMA samt sbit-instruktionen. }
    \label{tab:fpga_usage}
\end{table}

\section{Slutsats}

\subsection{Analys av prestanda}
Prestandan för set bit-instruktionen beror uteslutande på hur många minnesaccesser som behöver göras
vid anropet. Vid upprepade anrop med storleken 2 görs en skrivning till minnet var fjärde anrop, vilket gör att
snitt-tiden blir lägre än i övriga fall. För storleken 8 görs en skrivning vid varje anrop, vilket
leder till värsta fallet på 13 klockcykler i medel. Vid skrivning av 0xFF görs internt två separata skrivningar,
vilket gör att detta tar exakt lika lång tid som skrivningar av storleken 16, nämligen 17 cykler.\\

En intressant slutsats vi kan dra av detta är att det tar ganska mycket overhead bara att utföra vår
instruktion. För storlek 2 görs fyra gånger färre minnesskrivningar jämfört med storlek 8, men tiden
för dessa skiljer sig bara med en tredjedel. På samma sätt görs dubbelt så många skrivningar för
storlek 16 som storlek 8 med samma data, men skillnaden dem emellan är mindre än en fjärdedel.\\

Instruktionen gjorde en hel del skillnad när vi använde den i JPEG-acceleratorn. Som synes i Tabell
\ref{tab:jpeg_sbit_performance} minskade tiden för att skriva data till minnet vid Huffman-kodningen
(emit\_bits) från 5,0 miljoner cykler till 1,3 miljoner cykler. Den totala tiden minskade
därigenom med nästan lika mycket, vilket gav en total prestandaökning med 16\%. Gruppen är dock
enig om att prestandaförbättring-per-timme-spenderad-i-Muxen-indexet troligen är väldigt lågt.\\


\subsection{Möjliga förbättringar}
Ett problem med vår nuvarande hårdvara är att vi alltid skickar en STALL-signal till CPU:n så fort vi
får in en set bit-instruktion. Detta är i väldigt många fall inte alls nödvändigt (särskilt som vi
bara skriver till minnet om vi har en hel byte att skicka). Detta är särskilt problematiskt i ett
operativsystem med multitasking - det blir omöjligt att göra saker parallelt om vi säger åt resten
av datorn att sluta arbeta så fort någon använder vår instruktion.\\

En möjlig förbättring skulle kunna vara att hårdvaran detekterar i vilka fall som vi behöver skicka
en STALL-signal och inte. Detta behöver i så fall ske med kombinatorik för att hinna skicka signalen
tillräckligt fort, i de fall där detta behövs. På så sätt skulle vi kunna lösa problemen med
dålig paralellism i operativsystem med multitaskning, samt få något bättre prestanda generellt,
på bekostnad av ganska lite hårdvara.\\

Om målet istället skulle vara att hitta en billigare lösning på bekostnad av prestanda, hade vi kunnat använda
ett skiftregister för att stegvis skifta in varje bit data i vårt register, istället för vår nuvarande
lösning som troligen realiseras med en stor samling multiplexrar och OR-grindar för att lyckas göra
detta på en klockcykel.\\

\section{Appendix: Källkod}


\end{document}
