% !TEX TS-program = pdflatex
% !TEX encoding = UTF-8 Unicode

\documentclass[a4paper]{article}

\usepackage[swedish]{babel}
\usepackage[T1]{fontenc}
\usepackage[utf8]{inputenc}
\usepackage[pdftex]{graphicx}
\usepackage{float}
\usepackage{fancyhdr}
\usepackage[toc,page]{appendix}
\usepackage{listings}
\usepackage{booktabs} % for much better looking tables
\usepackage{array} % for better arrays (eg matrices) in maths
\usepackage{paralist} % very flexible & customisable lists (eg. enumerate/itemize, etc.)
\usepackage{verbatim} % adds environment for commenting out blocks of text & for better verbatim
\usepackage{subfig} % make it possible to include more than one captioned figure/table in a single float

\def\changemargin#1#2{\list{}{\rightmargin#2\leftmargin#1}\item[]}
\let\endchangemargin=\endlist

%%% HEADERS & FOOTERS
\author{Jonathan Karlsson, Niclas Olofsson, Paul Nedstrand\\jonka293, nicol271, paune415\\Grupp 2}
\pagestyle{fancy} % options: empty , plain , fancy
\renewcommand{\headrulewidth}{1pt} % customise the layout...
\fancyhead[LO,LE]{Jonathan, Niclas, Paul\\Rapport lab 2-3}
\lfoot{}\cfoot{\thepage}\rfoot{}

%%%% SECTION TITLE APPEARANCE
%\usepackage{sectsty}
%\allsectionsfont{\sffamily\mdseries\upshape} % (See the fntguide.pdf for font help)
%% (This matches ConTeXt defaults)
%
%%%% ToC (table of contents) APPEARANCE
%\usepackage[nottoc,notlof,notlot]{tocbibind} % Put the bibliography in the ToC
%\usepackage[titles,subfigure]{tocloft} % Alter the style of the Table of Contents
%\renewcommand{\cftsecfont}{\rmfamily\mdseries\upshape}
%\renewcommand{\cftsecpagefont}{\rmfamily\mdseries\upshape} % No bold!

%%% END Article customizations

%%% The "real" document content comes below...

\title{Rapport lab 2-3\\ \vspace{2 mm} {\large TSEA44}}
%\date{} % Activate to display a given date or no date (if empty),
         % otherwise the current date is printed

\begin{document}
\maketitle

\newpage

\tableofcontents

\newpage
\section{Labb 2}
\subsection{Introduktion}

Syftet med denna laboration var att konstruera en accelerator för jpeg-
komprimering från raw-filer, i hårdvara. Detta löstes med hjälp av
programmeringsspråket system verilog och genom att bygga vidare på det
existerande skelettet som gavs med uppgiften. I första delen (labb2) så
har vi en dator som styr själva acceleratorn och skickar all data via en
bus för att sedan behandlas och lagras i ett blockminne där datorn sedan
får läsa av resultatet. Ett register används för att starta grunkan och
sedan läsa av att resultatet finns att hämta.

Det jpeg-acceleratorn gör är att ta ett block med 8x8 pixlar, skicka in
det i en givet DCT maskin som gör komprimeringar av pixlarna, sedan
transponeras de 8x8 pixlarna i ett minne och skickas tillbaka till
DCT\rq{}n för att gå igenom en gång till. Alla pixlar multipliceras nu
med hårdkodade reciprokaler och läggs i ett utminne.

Datorn kan nu läsa av minnet, transponera och spara ner till en jpeg fil
som kan öppnas på vanligt sätt.

\subsection{Tillståndsgraf och arkitektur}

En vidare hjälp är att den DCT maskin som gör beräkningar är redan given
så problemet som skall lösas innebär att skapa signaler för att styra
DCT\rq{}n, ett blockminne för indata och ett för utdata, ett minne där
vi transponerar samt en maskin som utför multiplikationerna med
reciprokalerna.

\begin{figure}[h]
\centering
\includegraphics[scale=0.5]{architecture.png}
\caption{Arkitektur för vår design}
\label{fig:architecture}
\end{figure}

Figur \ref{fig:architecture} visar den arkitektur som konstruerades för
att lösa uppgiften. All data som kommer in från bussen läggs in i minnet
och vi väntar sedan på en startsignal från datorn. Då används räknaren
tillsammans med en vippa för att skicka in data till DCT\rq{}n för att
sedan skriva in 8x8 pixlar till transponeringsminnet, när vi skrivit
klart där så läser vi av kolumnerna (och får därigenom transponeringen),
skickar in till DCT\rq{}n igen.

Ett problem vi hade så här långt var klockningen, när vi skickar in data
från inminnet så måste vi göra det 8 pixlar åt gången, men i minnet
finns bara 4 pixlar per rad, så vi måste läsa två rader (därav vippan),
vidare måste vi klocka ner DCT\rq{}n en klockcykel för att den ska
\lq\lq{}hinna med\rq\rq{}. Läsningen från transponeringen var inget
problem, däremot när pixlarna kommer ut den andra gången måste
klockningen anpassas till Q2-maskinen då den tar två pixlar per
klockcykel så DCT\rq{}n måste gå ytterligare långsammare för att detta
ska fungera.

\begin{figure}[h]
\centering
\includegraphics[width=340px]{states.png}
\caption{Tillståndsgraf för vår design}
\label{fig:state}
\end{figure}

Figur \ref{fig:state} visar en tillståndsgraf över maskinen och dess
procedur för att komprimera raw-filer till jpeg. Tillståndsgrafen följer
precis det schema som arkitekturen i figur \ref{fig:architecture} också
visar.

\subsection{Vidare frågor man kan ställa sig}
\begin{itemize}
	\item Hur vi verifierade att vår DCT-hårdvara fungerade korrekt?
\end{itemize}

Vi använde oss av MATLAB för att räkna ut ett antal exempel som vi kunde
simulera i ModelSim, genom att jämföra vad MATLAB gav för resultat med
det som låg i utminnet efter en simulering lyckades vi bekräfta att
acceleratorn hade gjort sitt jobb.

\begin{itemize}
	\item Är storleken på DCT hårdvaran jämförbar med den förbättrade prestandan?
\end{itemize}

Hela grunkan tar upp en hel del hårdvara och på grund av att det är så
mycket kommunikation, vi måste klocka DCT\rq{}n annorlunda och att Q2 är
så långsam så är det frågan om det verkligen är värt det. I de kommande
laborationerna gör vi en del förbättringar på systemet som gör att det
är mer värdefullt.

\begin{itemize}
	\item Vidare optimeringar?
\end{itemize}

När man använder sig av den algoritm som gjorts i detta projekt så ser
man att nästan alla data hamnar i ett sicksackmönster i början av
minnet, så om man skulle läsa av minnet i ett sicksackmönster skulle man
spara in en hel del kommunikation eftersom det mestadels är nollor i
slutet. Man skulle kunna ha ett förutbestämt mönster för hur man läser
av minnet där man börjar i det övre vänstra hörnet och sedan förflyttar
sig sicksack, under tiden så skriver man ner alla siffror man stöter på
och hur många siffror det råkar vara. Stöter man på 3 nollor så
antecknar man det och sedan mot slutet bör det bli en hel del nollor.

Det man i så fall vinner är att man slipper skicka iväg alla siffror
utan man skickar bara iväg i vilken ordning man hittade dem och hur
många så de sista nollorna skickas som en nolla och antalet nollor
vilket är en mycket kortare sändning än att skicka en hel drös med
nollor. Detta skulle inte vara jättekomplicerat att implementera men
skulle förstås ta upp mer hårdvara, man skulle dock tjäna en hel del
kommunikation på systemet så det skulle nog vara värt det om man hade
höga krav på hastighet.

När man skriver in pixlarna till sin jpegfil skriver man bara tillbaka
siffrorna i samma ordning som man läste dem i, vilket skulle vara lika
svårt som när man läste dem i första läget. Läsning och skrivning blir
så klart mer komplicerat men det vinner man tillbaka flera gånger om
tack vare sparad kommunikation.

\section{Labb 3}
\subsection{Introduktion}

Designen från labb 2 visas i figur \ref{fig:architecture} och innebär
att datorn måste engagera sig en hel del och en stor flaskhals för
prestandan ligger i bussen som används. Dels måste datorn läsa ur alla
pixlar från minnet, via bussen, sedan skicka iväg pixlarna igen via
bussen till jpeg-acceleratorn för att sedan skicka en startsignal för
varje 8x8 pixlar. När acceleratorn är färdig måste datorn läsa av det i
ett register varpå datorn läser av minnet i acceleratorn och skickar
ännu mer pixlar genom bussen.

Den flaskhalsen skulle man kunna spara in en del på genom att
acceleratorn själv hämtar sin data från minnet istället för att gå genom
datorn, där gör den ändå ingen nytta. Så idén är att ha en ytterligare
design som hämtar data och skickar in till acceleratorn, sätter igång
den självmant. Datorn måste fortfarande läsa av registret för att
kontrollera att acceleratorn är färdig samt hämta de färdiga pixlarna,
men bussen besparas ändå en hel del data med denna lösning.

\subsection{Design}
\begin{figure}[h]
\centering
\includegraphics[width=280px]{fsmlab3.png}
\caption{Tillståndsgraf för vår utökade design}
\label{fig:state}
\end{figure}

För att lösa uppgiften så implementerade vi tillståndsgrafen i figur
\ref{fig:state2} ganska rakt av med hjälp av en DMA. Maskinen börjar i
\lq\lq{}Idle\rq\rq{}, får en startsignal och börjar hämta block med 8x8
pixlar från minnet, skickar iväg pixlarna till acceleratorn och inväntar
sedan en signal att den är färdig. Då hämtar den ett nytt block och
håller på så tills hela bilden är bearbetad. \lq\lq{}Releasebus\rq\rq{}
är till för att då och då släppa bussen så att någon annan i maskinen
kan få skicka saker via bussen, annars hade DMA\rq{}n tagit beslag på
bussen väldigt länge och inget annat skulle hända i datorn medans
DMA\rq{}n hämtade data.

\begin{figure}[h]
\centering
\includegraphics[width=280px]{architecturelab3.png}
\caption{Arkitektur för vår utökade design}
\label{fig:arch3}
\end{figure}

Figur \ref{fig:arch3} visar hur DMA\rq{}n är kopplad till acceleratorn
och hur den kontrollerar vilka signaler som kommer in, antingen från
bussen eller från DMA\rq{}n. Jämför med figur \ref{fig:architecture} för
att se resten av designen. I DMA modulen finns bara register för att
hålla koll på vilket tillstånd maskinen befinner sig i, när vissa
villkor är uppfyllda byts tillståndet och korrekta åtgärder vidtas. En
räknare håller koll på hur många pixlar som har lästs in, när det har
lästs 4 pixlar skickas de till acceleratorn, \lq\lq{}Waitready\rq\rq{}.
Detta görs om och om igen tills hela bilden är färdig då vi istället går
till \lq\lq{}Waitreadylast\rq\rq{} och sedan tillbaka till
\lq\lq{}Idle\rq\rq{}.

Verifiering av hårdvaran skedde genom att köra de tidigare testerna i
modelsim och när rätt signaler lades på så testade vi att konvertera en
raw fil till jpeg vilket fungerade. Mjukvaran som kallade på DMA\rq{}n
skickade en startsignal och väntade sedan på att acceleratorn skulle
sätta färdigbiten i dess statusregister, därefter läste mjukvaran av
utminnet och skickade en fortsätt-signal, detta skedde tills bilden var
klar.

Det som skickas på bussen är således en startsignal från mjukvaran, 8x8
pixlar till DMA\rq{}n, varpå mjukvaran kontinuerligt läser av
statusregistret för att se när blocket är färdigt, varpå mjukvaran kan
läsa av de 8x8 pixlarna. All den kommunikationen görs tills bilden är
färdig.

\section{Resultat}

\subsection{Prestanda}

\begin{table}[H]
    \centering
    \begin{tabular}{l l}

        Beskrivning & Antal klockcykler\\
        \hline
        Main program  & 33 902 171 \\
        Init  &  9 850 924 \\
        Encode\_image  & 24 051 247 \\
        Forward\_DCT  & 8 317 191 \\
        Copy  & 1 518 825 \\
        DCT kernel  & 0 \\
        Quantization  & 6 798 366 \\
        Huffman encoding  & 15 246 407 \\
        Emit\_bits  & 7 904 811 \\
    \end{tabular}
    \caption{ Prestanda för JPEG-acceleratorn }
    \label{tab:jpeg_performance_2}
\end{table}
\begin{table}[H]
    \centering
    \begin{tabular}{l l}

        Beskrivning & Antal klockcykler\\
        \hline
        Main program  & 25 847 242 \\
        Init  & 6 600 044 \\
        Encode\_image  & 19 247 198 \\
        Forward\_DCT  & 6 489 189 \\
        Copy  & 0 \\
        DCT kernel  & 0 \\
        Quantization  & 6 489 189 \\
        Huffman encoding  & 12 226 950 \\
        Emit\_bits  & 4 983 905 \\
    \end{tabular}
    \caption{ Prestanda för JPEG-acceleratorn med inkopplad DMA }
    \label{tab:jpeg_performance_3}
\end{table}

Tabell \ref{tab:jpeg_performance_2} visar prestandan för labb2 och
tabell \ref{tab:jpeg_performance_3} för labb3, i \lq\lq{}Copy\rq\rq{}
har prestandan sjunkit med 1,5 miljoner klockcykler! Totalt gick labb3 8
miljoner klockcykler.

\subsection{FPGA-användning}
\begin{table}[H]
    \centering
    \begin{tabular}{l l l}
        Flip Flops   &      7 273 out of   46080  & 15\% \\
        4 input LUTs &      11 485 out of  46080  & 24\% \\
        MULT18X18s   &      19 out of 120  &   15\% \\
        RAMB16s      &      42 out of 120  &   35\% \\
    \end{tabular}
    \caption{ De mest intressanta delarna ur syntes-rapporten för JPEG-acceleratorn utan DMA. }
    \label{tab:fpga_usage2}
\end{table}
\begin{table}[H]
    \centering
    \begin{tabular}{l l l}
        Flip Flops   &      7 408 out of   46080  & 16\% \\
        4 input LUTs &      11 986 out of  46080  & 26\% \\
        MULT18X18s   &      19 out of 120  &   15\% \\
        RAMB16s      &      42 out of 120  &   35\% \\
    \end{tabular}
    \caption{ De mest intressanta delarna ur syntes-rapporten för JPEG-acceleratorn med DMA. }
    \label{tab:fpga_usage3}
\end{table}

Tabell \ref{tab:fpga_usage2} visar en del ur syntetiseringsrapporten för acceleratorn, jämför man med tabell \ref{tab:fpga_usage3} så ser man att DMA\rq{}n inte gav så stor skillnad i storlek, men gav bra prestandaförbättring så det var värt att implementera den.

\section{Filer}
\begin{itemize}
        \item [jpegtop.sv] Mestadels av designen skedde här, våra kontrollsignaler ligger här tillsammans med flera räknare och minnen.
        \item [transpose.sv] Transponeringsminnet fick en egen fil med en kolumnräknare och en radräknare för att hålla koll på vart grunkan håller på att läsa respektive skriva.
        \item [q2.sv] Denna sköter multipliceringen med reciprokalerna och skickar sedan resultatet till utminnet.
        \item [jpegdma.sv] Designen för den andra biten där jpeg-acceleratorn själv läser av informationen i minnet.
\end{itemize}


\newpage
\begin{appendices}
\begin{changemargin}{-3cm}{-3cm}
\section{jpegtop.sv}
\lstinputlisting[language=Verilog,caption={Kod för labb 2},label=jpeg_top]{jpeg_top.sv}

\section{transpose.sv}
\lstinputlisting[language=Verilog,caption={Kod för transponeringen},label=selection-sort]{transpose.sv}

\section{Q2.sv}
\lstinputlisting[language=Verilog,caption={Kod för kvantiseringen},label=q2]{q2.sv}

\section{jpegdma.sv}
\lstinputlisting[language=Verilog,caption={Kod för labb 3},label=jpeg_dma]{jpeg_dma.sv}
\end{changemargin}
\end{appendices}


\end{document}
