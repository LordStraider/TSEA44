% !TEX TS-program = pdflatex
% !TEX encoding = UTF-8 Unicode

\documentclass[a4paper]{article}

\usepackage[swedish]{babel}
\usepackage[T1]{fontenc}
\usepackage[utf8]{inputenc}
\usepackage[pdftex]{graphicx}
\usepackage{float}
\usepackage{fancyhdr}


\usepackage{booktabs} % for much better looking tables
\usepackage{array} % for better arrays (eg matrices) in maths
\usepackage{paralist} % very flexible & customisable lists (eg. enumerate/itemize, etc.)
\usepackage{verbatim} % adds environment for commenting out blocks of text & for better verbatim
\usepackage{subfig} % make it possible to include more than one captioned figure/table in a single float


%%% HEADERS & FOOTERS
\author{Jonathan Karlsson, Niclas Olofsson, Paul Nedstrand\\jonka293, nicol, paune\\Grupp 2}
\pagestyle{fancy} % options: empty , plain , fancy
\renewcommand{\headrulewidth}{1pt} % customise the layout...
\fancyhead[LO,LE]{Jonathan, Niclas, Paul\\Rapportmall}
\lfoot{}\cfoot{\thepage}\rfoot{}

%%%% SECTION TITLE APPEARANCE
%\usepackage{sectsty}
%\allsectionsfont{\sffamily\mdseries\upshape} % (See the fntguide.pdf for font help)
%% (This matches ConTeXt defaults)
%
%%%% ToC (table of contents) APPEARANCE
%\usepackage[nottoc,notlof,notlot]{tocbibind} % Put the bibliography in the ToC
%\usepackage[titles,subfigure]{tocloft} % Alter the style of the Table of Contents
%\renewcommand{\cftsecfont}{\rmfamily\mdseries\upshape}
%\renewcommand{\cftsecpagefont}{\rmfamily\mdseries\upshape} % No bold!

%%% END Article customizations

%%% The "real" document content comes below...

\title{Rapportmall\\ \vspace{2 mm} {\large TSEA44}}
%\date{} % Activate to display a given date or no date (if empty),
         % otherwise the current date is printed 

\begin{document}
\maketitle

\section{What to Include in the Lab Report 1}

The lab report should contain all source code that you have written. (The source code should of course be commented.) We would also like you to include a block diagram of your hardware. If you have written any FSM you should include a state diagram graph of the FSM.
We would also like you to discuss the following questions:
\begin{itemize}
\item  How did you verify that your computer hardware worked?
\item  What is the performance of the 2D DCT software? (Try it with and without caches.)
\item How much of the FPGA is used by our design?
\end{itemize}
And of course, the normal parts of a lab report such as a table of contents, an intro- duction, a conclusion, etc. The source code that you have written should be included in appendices and referred to from the main document.

\section{What to Include in the Lab Report 2}
The lab report should contain all source code that you have written. (The source code should of course be commented.) We would also like you to include a block diagram of your hardware. If you have written any FSM you should include a state diagram graph of the FSM.
We would also like you to discuss the following questions in detail somewhere in your lab report.
\begin{itemize}
\item How does your 2D DCT hardware work?
\item How did you verify that your 2D DCT hardware works correctly?
\item What is the performance with and without the 2D DCT hardware? This should include measurements of both the 2D DCT kernel and the entire application.
\item A timestamp diagram.
\item How much of the FPGA is used by the 2D DCT hardware?
\item How much is the 2D DCT hardware used while encoding an image in jpegtest?
\item Isthesizeofthe2DDCThardwarejustifiedbytheperformanceimprovements?
\item What would be required in order to implement more functionality like zigzag addressing in the 2D DCT hardware module? Would it be difficult to modify jpegfiles to take advantage of such optimizations?
And of course, the normal parts of a lab report such as a table of contents, an intro- duction, a conclusion, etc. The source code that you have written should be included in appendices and referred to from the main document.
\end{itemize}


\section{What to Include in the Lab Report 3}
The lab report should contain all source code that you have written. (The source code should of course be commented.) We would also like you to include a block diagram of your hardware. If you have written any FSM you should include a state diagram graph of the FSM.
We would also like you to discuss the following questions in detail somewhere in your lab report1:
\begin{itemize}
\item How does your hardware work?
\item How did you verify that your hardware worked?
\item How did you modify the software?
\item A timing diagram.
\item What is the utilization of your accelerator?
\item What is the performance of jpegtest with DMA enabled?
\item How long does it take (on average) to read a macroblock into the DCT acceler- ator via DMA?
\item How much is the wishbone bus used by the DMA unit and how much is the bus used by the CPU?
And of course, the normal parts of a lab report such as a table of contents, an intro- duction, a conclusion, etc. The source code that you have written should be included in appendices and referred to from the main document.
\end{itemize}


\begin{table}[h]
	\centering
 	\begin{tabular}{l l l l}
		stakeholders 			& change1 	& change2  	& change3 \\
		stakeholder1 			& + / -		& + / -		& + / - \\
		stakeholder2 			& + / -		& + / -		& + / - \\
		stakeholder3 			& + / -		& + / -		& + / - \\
	\end{tabular}
	
	\caption{Shows how the different stakeholders are affected by each change.}
	\label{tab:table1}
\end{table}

\begin{table}[h]
	\centering
 	\begin{tabular}{l l l}
	    change1 & description & description \\
	    change2 & description & description \\
	    change3 & description & description \\
	\end{tabular}
	
	\caption{Describes the changes and why they effect the stakeholders.}
	\label{tab:table1}
\end{table}


\section{What to Include in the Lab Report 4}
The lab report should contain all source code that you have written. (The source code should of course be commented.) We would also like you to include a block diagram of your hardware. If you have written any FSM you should include a state graph of the FSM.
We would also like you to discuss the following questions in detail somewhere in your lab report:
\begin{itemize}
\item How does your hardware work?
\item How did you verify that your set bit hardware worked?
\item What is the performance with and without the set bit hardware? This should include measurements of both the entire application and the set bit instruction by itself, assuming good code in a software implementation (take a look at how the software solution in jpegfiles).
\item How much of the FPGA does your hardware use?
\item How would your design change if you had to achieve even higher speed using
more hardware?
\item How would your design change if you had to use less hardware at the cost of a slower solution?
\item Whataretheproblemswithusingyournewhardwareinamultitaskingoperating system? How can the problem(s) be solved?
\item What is the performance of your final system?
\item What was the hardest problems you encountered during the entire lab course?
And if you want to, we would appreciate some comments on the following ques- tions, either in the lab report or by some other means of communications:
\item What did you think of the TSEA44 course? • What was good?
70
CHAPTER6. LABTASK4-CUSTOMINSTRUCTIONS
\item What was bad?
\item What can we improve for the next year?
\item Do you have any other ideas for this course?
\item Did you feel that you learned anything of value? • Any other comments you may have.
\item A rough estimation of time spent on the lab tasks.
And of course, the normal parts of a lab report such as a table of contents, an intro- duction, a conclusion, etc. The source code that you have written should be included in appendices and referred to from the main document.
\end{itemize}


\end{document}
